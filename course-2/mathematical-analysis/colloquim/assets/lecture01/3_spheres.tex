% \documentclass[a4paper]{article}
% \usepackage[english,russian]{babel}
%
%
% \usepackage{tikz}
% \usetikzlibrary{arrows.meta}
% \begin{document}

\begin{tikzpicture}[
    set/.style={dashed, thick},
    arrow/.style={-{Stealth[scale=1.2]}, thick}
]

\draw (-4, 0) circle (40pt);
\draw[latex-latex, line width=1pt] (-3, -1) -- (-5, 1);
\node[rotate=-45] at (0.2, 0.2) {d};

\draw[set] (0, 0) circle (40pt);
\draw[latex-latex, line width=1pt] (1, -1) -- (-1, 1);
\node[rotate=-45] at (-3.8, 0.2) {d};

\fill (3, 1.3) circle (1.75pt);
\fill (4, 0) circle (1.75pt);
\fill (5, -1.3) circle (1.75pt);
\draw[latex-latex, line width=1pt] (5.15, -1.2) -- (3.15, 1.4);
\draw[Bar-Bar, line width=0.7pt] (5.15, -1.2) -- (3.15, 1.4);
\node[rotate=-55] at (4.4, 0.2) {d};

\node[align=center] at (0,-2) {Пример диаметра для разных ограниченных множеств(Для всех трёх он равен $d$)};

\end{tikzpicture}


% \end{document}
