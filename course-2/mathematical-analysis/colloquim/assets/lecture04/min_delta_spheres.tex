%\documentclass[tikz,border=3.14mm]{standalone}
%\usepackage[english,russian]{babel} % локализация и переносы
%\usetikzlibrary{arrows.meta}

% \begin{document}
\begin{tikzpicture}[
    set/.style={dashed, thick},
    arrow/.style={-{Stealth[scale=1.2]}, thick}
]

% Открытое множество (клякса)
\draw[thin] (0,0) to[out=30,in=150] (2,1.0)
           to[out=-30,in=60] (3.5,-1.5)
           to[out=-120,in=-30] (1,-2)
           to[out=120,in=-120] (0.5,-1)
           to[out=60,in=180] cycle;

\path (-2, -0.5) node (point0) {};

\fill (point0) circle (1.5pt);
\node[right] at (-2.2,-0.2) {$x_0$};

\draw[set][red!] (point0) circle (35pt);
\node[right] at (-2.8,-2.1) {$B_{\frac{\delta}{2}}(x_0)$};


% Стрелка с подписью
\draw[arrow] (3.5,0.8) -- (3.2,0.1);
\node[right] at (3.3,1) {$K$};

\path (1, 0.5) node (point_x1) {};
\draw[set] (point0) to (point_x1);
\draw[set](point_x1) circle (44pt);
\fill (point_x1) circle (1.5pt);
\node[right] at (point_x1) {$x_1$};
\node[right] at (2,1.8) {$B_{\frac{\delta_1}{2}}(x_1)$};
\fill[blue!] (-0.5, -0) circle (2pt);
\draw[blue!] (-0.65, 0.47) to (-0.34, -0.47);


\path (1.2, -1.8) node (point_x2) {};
\draw[set] (point0) to (point_x2);
\draw[set](point_x2) circle (48pt);
\fill (point_x2) circle (1.5pt);
\node[right] at (point_x2) {$x_2$};
\node[right] at (2.5,-2.8) {$B_{\frac{\delta_2}{2}}(x_2)$};
\fill[blue!] (-0.4, -1.15) circle (2pt);
\draw[blue!] (-0.21, -0.68) to (-0.58, -1.61);

\node[align=center] at (0.4,-4.1) {Пример как мы строим $B_{\frac{\delta}{2}}$ вокруг точки $x_0$. \\ Синие точки - середины отрезков на которых они лежат};

\end{tikzpicture}
%\end{document}
