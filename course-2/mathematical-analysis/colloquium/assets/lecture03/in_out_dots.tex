\begin{tikzpicture}[
    set/.style={dashed, thick},
    arrow/.style={-{Stealth[scale=1.2]}, thick}
]

\begin{scope}[scale=1.8]


\draw[set, red!] (-1.7,0) to[out=130,in=70] (-2.9,0.3)
           to[out=-100,in=60] (-2.8,-1.)
           to[out=-120,in=-100] (-1.8,-1.2)
           to[out=60,in=-50] cycle;
\fill [pattern = {Lines[angle = -45, line width = 1pt, distance = 10pt]},
    pattern color = blue]
        (-1.7,0) to[out=130,in=70] (-2.9,0.3)
           to[out=-100,in=60] (-2.8,-1.)
           to[out=-120,in=-100] (-1.8,-1.2)
           to[out=60,in=-50] cycle;


\draw[thin, red!] (0,0) to[out=30,in=150] (0.5,0.3)
           to[out=-30,in=60] (1.,-1.)
           to[out=-120,in=-30] (0.3,-1.2)
           to[out=120,in=-120] (-0.3,-0.2)
           to[out=60,in=180] cycle;
\fill [pattern = {Lines[angle = -45, line width = 1pt, distance = 10pt]},
    pattern color = blue]
        (0,0) to[out=30,in=150] (0.5,0.3)
           to[out=-30,in=60] (1.,-1.)
           to[out=-120,in=-30] (0.3,-1.2)
           to[out=120,in=-120] (-0.3,-0.2)
           to[out=60,in=180] cycle;


\fill[green!] (-1, 0.4) circle(1.25pt);
\fill[green!] (-0.4, -0.7) circle(1.25pt);
\fill[green!] (-1.2, -0.5) circle(1.25pt);
\fill[green!] (-0.9, -1.2) circle(1.25pt);
\fill[green!] (-0.7, -0.1) circle(1.25pt);
\fill[green!] (-3.2, -0.8) circle(1.25pt);
\fill[green!] (-3.4, 0.1) circle(1.25pt);
\fill[green!] (1.5, -0.3) circle(1.25pt);

\end{scope}


\node[align=center] at (-1.5,-4) {Синяя область - пример множества внутренних точек \\ Зеленые - пример внешних (изображена часть точек) \\ Красные границы -  пример множества граничных точек (пример $\partial M$)};


\end{tikzpicture}
