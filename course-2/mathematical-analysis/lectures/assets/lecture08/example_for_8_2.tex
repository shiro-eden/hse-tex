
\pgfplotsset{compat=1.17}
\begin{tikzpicture}[scale=0.5]
    \begin{axis}[
        view={30}{30},
        xlabel={$x$},
        ylabel={$y$},
        zlabel={$z$},
        grid=major,
        colormap/viridis,
        width=10cm,
        height=10cm,
        zmin=0, zmax=4,
        xmin=-2, xmax=2,
        ymin=-2, ymax=2,
        ]

        % Воронка, направленная вниз
        \addplot3[
        surf,
        domain=-2:2,
        domain y=-2:2,
        samples=40,
        opacity=0.7,
        ] {4 - 1/(0.3 + x^2 + y^2)};

        % Плоскость сечения (x = 0.5)
        \addplot3[
        surf,
        domain=-2:2,
        domain y=0:4,
        samples=2,
        opacity=0.3,
        color=gray,
        ] (0.5, x, y);

        % Линия пересечения
        \addplot3[
        thick,
        red,
        domain=-1.65:1.85,
        samples=30,
        smooth,
        ] (0.5, x, {4 - 1/(0.3 + 0.25 + x^2)});

        \addplot3[
        only marks,
        mark=*,
        mark size=2,
        color=blue
        ] coordinates {(0.5, 0, {4 - 1/(0.3 + 0.25 + 0^2)})};

        \addplot3[
        only marks,
        mark=*,
        mark size=2,
        color=blue
        ] coordinates {(0, 0, {4 - 1 / 0.3})};


    \end{axis}

    \node[align=center] at (4, -1.5) {Если зафиксировать какой-то $x$ и искать $\inf$ по $y$ \\ то он всегда будет больше, чем $\inf$ на всей области};
\end{tikzpicture}
