 % локализация и переносы
% \documentclass[tikz,border=10pt]{standalone}
% \usepackage[utf8]{inputenc}         % кодировка исходного текста
% \usepackage[english,russian]{babel}
\usetikzlibrary{decorations.pathmorphing}
% \usetikzlibrary{arrows.meta, patterns, patterns.meta}

% \begin{document}
\begin{tikzpicture}[scale=2, every node/.style={inner sep=0,outer sep=0},
    set/.style={dashed, thick},
    arrow/.style={-{Stealth[scale=1.2]}, thick}]

     \begin{scope}[shift={(0.3, 0.3)}]

        \draw[blue, thick] (0,0) -- (1,0);
        \draw[blue, thick] (0,0) -- (0,1);
        \draw[blue, thick] (0,1) -- (1,1);
        \draw[blue, thick] (1,0) -- (1,1);
    \end{scope}

     \begin{scope}

        \draw[->] (0, 0) -- (1.5, 0) node[right] {$x$};
        \draw[->] (0, 0) -- (0, 1.5) node[above] {$y$};
    \end{scope}



    \draw[arrow] (1.8, 0.5) -- (2.6, 0.5);
    \node at (2.2, 0.35) {$\varphi$};

    \draw[arrow] (2.6, 1) -- (1.8, 1);
    \node at (2.3, 1.15) {$\varphi^{-1}$};




    \begin{scope}[shift={(3.2, 0.3)}, rotate=10]

        \draw[blue, thick] (0,0) .. controls (0.5, 0.1) .. (1,0);
        \draw[blue, thick] (0,0) .. controls (0.1, 0.5) .. (0,1);
        \draw[blue, thick] (0,1) .. controls (0.5, 1.1) .. (1,1);
        \draw[blue, thick] (1,0) .. controls (1.1, 0.5) .. (1,1);
    \end{scope}


     \begin{scope}[shift={(3, 0)}, rotate=10]

        \draw[->] (0, 0) -- (1.5, 0) node[right] {$u$};
        \draw[->] (0, 0) -- (0, 1.5) node[above] {$v$};
    \end{scope}

     \begin{scope}[shift={(3, 0)}, opacity=0.3]

        \draw[->] (0, 0) -- (1.5, 0) node[right] {$x$};
        \draw[->] (0, 0) -- (0, 1.5) node[above] {$y$};
    \end{scope}


    \node[align=center] at (2.3, -0.3) {Отображение допустимого множества в новые координаты};

\end{tikzpicture}
% \end{document}


