\section{Лекция 1}

\subsection{Брус. Мера бруса}

\definition Замкнутый брус (координатный промежуток) в $\mathbb{R}^n$ — множество, описываемое как



\begin{equation*}
\begin{aligned}
    I&=\{x\in\mathbb{R}^n\ |\ a_i\leq x_i\leq b_i,\ i\in\{1,n\}\}\\
    &=\left[a_1,b_1\right]\times\ldots\times\left[a_n,b_n\right]
\end{aligned}
\end{equation*}

\comment $I=\{a_1,b_1\}\times\ldots\times\{a_n,b_n\}$, где $\{a_i, b_i\}$ может быть отрезком, интервалом и т.д.

\begin{center}
    % \documentclass[a4paper]{article}
% \usepackage[english,russian]{babel}
%
%
% \usepackage{tikz}
\usetikzlibrary{arrows.meta, 3d, perspective}
% \begin{document}

\begin{tikzpicture}[
    set/.style={dashed, thick},
    arrow/.style={-{Stealth[scale=1.2]}, thick}
]


\def\a{2} % длина
\def\b{2} % ширина
\def\c{2} % высота

    % Координаты вершин куба
\coordinate (A) at (0,0,0);
\coordinate (B) at (\a,0,0);
\coordinate (C) at (\a,\b,0);
\coordinate (D) at (0,\b,0);
\coordinate (E) at (0,0,\c);
\coordinate (F) at (\a,0,\c);
\coordinate (G) at (\a,\b,\c);
\coordinate (H) at (0,\b,\c);

    % Рисуем невидимые ребра (пунктиром)
\draw[dashed] (A) -- (D);
\draw[dashed] (A) -- (B);
\draw[dashed] (A) -- (E);

    % Рисуем видимые ребра
\draw (B) -- (C) -- (G) -- (F) -- (B);
\draw (D) -- (H);
\draw (C) -- (D);
\draw (F) -- (G) -- (H) -- (E) -- (F);

\draw[<->,thick] (0.2, 0, \c + 0.5) -- node[below] {$[a_1 ; b_1]$} (\a + 0.1, 0, \c + 0.5);

\draw[<->,thick] (\a + 0.2, 0.1, 0) -- node[below,rotate=90] {$[a_3 ; b_3]$} (\a + 0.2, \b, 0);

\draw[<->,thick] (\a + 0.2, 0, \c) -- node[below,sloped] {$[a_2 ; b_2]$} (\a + 0.2, 0, \c - 1.8);


\draw (-5, -0.8) rectangle (-3, 1.2);
\draw[<->,thick] (-5, -1) to (-3, -1);
\node at (-4, -1.3) {$[a_1 ; b_1]$};
\draw[<->,thick] (-2.8, 1.2) to (-2.8, -0.8);
\node[rotate=90] at (-2.4, 0.2) {$[a_2 ; b_2]$};

\draw[|-|] (-9, 0.2) to (-7, 0.2);
\draw[<->,thick] (-9, -0.1) to (-7, -0.1);
\node at (-8, -0.4) {$[a_1 ; b_1]$};



\node[align=center] at (-4,-2) {Пример брусов размерности с 1 по 3};

\end{tikzpicture}


% \end{document}

\end{center}

\definition Мера бруса — его объём:

\begin{equation*}
    \begin{aligned}
        \mu(I)&=|I|
        =\prod_{i=1}^{n} (b_i-a_i)
    \end{aligned}
\end{equation*}

\subsection{Свойства меры бруса в $\R^n$}

\begin{enumerate}
    \item \textbf{Однородность:} $\mu(I_{\lambda a,\lambda b})=\lambda^n\cdot\mu(I_{a,b})$, где $\lambda\geq
    0$
    \item \textbf{Аддитивность:} Пусть $I, I_1, \ldots, I_k$ — брусы
    
    Тогда, если $\forall i, j\, I_i, I_j$ не имеют общих внтренних точек, и $\displaystyle\bigcup_{i=1}^kI_i = I$, то
    $$|I| = \sum_{i=1}^k|I_i|$$
    \item \textbf{Монотонность}: Пусть $I$ — брус, покрытый конечной системой брусов, то есть $I\subset \displaystyle\bigcup_{i=1}^kI_i$, тогда
    $$|I| < \sum_{i=1}^k|I_i|$$
\end{enumerate}

\subsection{Разбиение бруса. Диаметр множества. Масштаб разбиения}

\definition \label{1.3} $I$ — замкнутый, невырожденный брус и $\displaystyle\bigcup_{i=1}^kI_i = I$, где $I_i$ попарно не имеют общих внутренних точек. Тогда набор $\T = \{\T\}_{i=1}^k$ называется разбиением бруса $I$

\definition \label{1.4} Диаметр произвольного ограниченного множества $M\subset\R^n$ будем называть 

\begin{equation*}
\begin{aligned}
    d(M) = \displaystyle\sup_{1\leq i\leq k}\|x-y\|,\text{ где}\\
    \|x-y\|=\sqrt{\sum_{i=1}^{n}\left(x_i-y_i\right)^2}
\end{aligned}
\end{equation*}

\begin{center}
    % \documentclass[a4paper]{article}
% \usepackage[english,russian]{babel}
%
%
% \usepackage{tikz}
% \usetikzlibrary{arrows.meta}
% \begin{document}

\begin{tikzpicture}[
    set/.style={dashed, thick},
    arrow/.style={-{Stealth[scale=1.2]}, thick}
]

\draw (-4, 0) circle (40pt);
\draw[latex-latex, line width=1pt] (-3, -1) -- (-5, 1);
\node[rotate=-45] at (0.2, 0.2) {d};

\draw[set] (0, 0) circle (40pt);
\draw[latex-latex, line width=1pt] (1, -1) -- (-1, 1);
\node[rotate=-45] at (-3.8, 0.2) {d};

\fill (3, 1.3) circle (1.75pt);
\fill (4, 0) circle (1.75pt);
\fill (5, -1.3) circle (1.75pt);
\draw[latex-latex, line width=1pt] (5.15, -1.2) -- (3.15, 1.4);
\draw[Bar-Bar, line width=0.7pt] (5.15, -1.2) -- (3.15, 1.4);
\node[rotate=-55] at (4.4, 0.2) {d};

\node[align=center] at (0,-2) {Пример диаметра для разных ограниченных множеств(Для всех трёх он равен $d$)};

\end{tikzpicture}


% \end{document}

\end{center}


\definition \label{1.5} Масштаб разбиения $\T=\{I_i\}_{i=1}^k$ — число $\lambda(\T) = \Delta_{\T} = \displaystyle\max_{1\le i\le k} d(I_i)$

\definition \label{1.6} Пусть $\forall\ I_i$ выбрана точка $\xi_i\in I_i$. Тогда, набор $\xi = \{\xi_i\}_{i=1}^k$ будем называть \textbf{отмеченными точками}

\definition \label{1.7} Размеченное разбиение — пара $(\T, \xi)$

\subsection{Интегральная сумма Римана. Интегрируемость по Риману}
Пусть $I$ — невырожденный, замкнутый брус, функция $f: I\rightarrow \R$ определена на $I$

\definition \label{1.8} Интегральная сумма Римана функции $f$ на $(\T, \xi)$ — величина
$$\sigma(f, \T, \xi) := \sum_{i=1}^kf(\xi_i)\cdot|I_i|$$

\definition \label{1.9} Функция $f$ интегрируема (по Риману) на замкнутом брусе $I$ ($f:I\rightarrow\R$), если 

\begin{equation*}
\begin{aligned}
    \exists A\in\R: \forall \varepsilon > 0\, \exists \delta > 0: \forall(\T, \xi): \Delta_{\T} < \delta:\\
    |\sigma(f, \T, \xi)| - A| < \varepsilon
\end{aligned}
\end{equation*}

Тогда 
$$A = \int\limits_If(x)\d{x} = \underset{I}{\int\ldots\int}f(x_1, \ldots, x_n)\d{x_1}\ldots \d{x_n}$$
Обозначение: $f\in\mathcal{R}(I)$

\begin{center}
    % \documentclass[tikz,border=5pt]{standalone}
% \usepackage{pgfplots}
\pgfplotsset{compat=newest}
% \usepackage[english,russian]{babel}

% \begin{document}
\begin{tikzpicture}
  \begin{axis}[
    view={120}{30}, % угол обзора
    axis lines=center,
    xlabel={$x$}, ylabel={$y$}, zlabel={$z$},
    ticks=none,
    domain=0:1, y domain=0:1,
    samples=31, samples y=31,
    colormap/viridis,
    zmax=0.8, zmin=0,
    xmin=0, xmax=1.2,
    ymin=0, ymax=1.2
  ]

  % поверхность f(x,y) = 0.4 + 0.2*sin(2*pi*x)*cos(2*pi*y)
  \addplot3[surf, opacity=0.7]
    {0.6 + 0.05*sin(deg(2*pi*x))*cos(deg(2*pi*y))};

  % сетка внизу
  \foreach \i in {0,0.2,...,1} {
    \addplot3[black, thin] coordinates {(\i,0,0) (\i,1,0)};
    \addplot3[black, thin] coordinates {(0,\i,0) (1,\i,0)};
  }

  % координаты выбранного квадратика
  \pgfmathsetmacro{\xA}{0.4}
  \pgfmathsetmacro{\xB}{0.6}
  \pgfmathsetmacro{\yA}{0.4}
  \pgfmathsetmacro{\yB}{0.6}

  % значения функции в углах
  \pgfmathsetmacro{\zA}{0.6 + 0.05*sin(2*pi*\xA r)*cos(2*pi*\yA r)}
  \pgfmathsetmacro{\zB}{0.6 + 0.05*sin(2*pi*\xB r)*cos(2*pi*\yA r)}
  \pgfmathsetmacro{\zC}{0.6 + 0.05*sin(2*pi*\xB r)*cos(2*pi*\yB r)}
  \pgfmathsetmacro{\zD}{0.6 + 0.05*sin(2*pi*\xA r)*cos(2*pi*\yB r)}

  % нижний квадратик (основание)
  \addplot3[fill=red, opacity=0.2]
    coordinates {(\xA,\yA,0) (\xB,\yA,0) (\xB,\yB,0) (\xA,\yB,0) (\xA,\yA,0)};

  % верхний квадратик (на поверхности)
  \addplot3[fill=red, opacity=0.3]
    coordinates {(\xA,\yA,\zA) (\xB,\yA,\zB) (\xB,\yB,\zC) (\xA,\yB,\zD) (\xA,\yA,\zA)};

  % вертикальные рёбра
  \addplot3[red, thick, opacity=0.3] coordinates {(\xA,\yA,0) (\xA,\yA,\zA)};
  \addplot3[red, thick, opacity=0.3] coordinates {(\xB,\yA,0) (\xB,\yA,\zB)};
  \addplot3[red, thick, opacity=0.3] coordinates {(\xB,\yB,0) (\xB,\yB,\zC)};
  \addplot3[red, thick, opacity=0.3] coordinates {(\xA,\yB,0) (\xA,\yB,\zD)};

  % точка внутри квадратика внизу
  \addplot3[only marks, mark=*, red] coordinates {(0.5,0.5,0)};

  % точка на поверхности
  \pgfmathsetmacro{\zP}{0.6 + 0.05*sin(2*pi*0.5 r)*cos(2*pi*0.5 r)}
  \addplot3[only marks, mark=*, red] coordinates {(0.5,0.5,\zP)};

  % соединение точек
  \addplot3[dashed, red] coordinates {(0.5,0.5,0) (0.5,0.5,\zP)};

  \end{axis}

  \node[align=center] at (4.2,0) {Пример интегрирования в $\R^2$ по определению};

\end{tikzpicture}
% \end{document}

\end{center}


\subsection{Пример константной функции}

Пусть у нас есть функция $f = \text{const}$
\begin{equation*}
\begin{aligned}
    \forall(\T, \xi):\ \sigma(f, \T, \xi)&= \sum_{i = 1}^k \text{const}\cdot|I_i|\\
    &= \text{const}\cdot|I| \Longrightarrow \int_I f(x)\d{x} = \text{const}\cdot|I|
    \end{aligned}
\end{equation*}

\subsection{Пример неинтегрируемая функция}

Имеется брус $I = [0, 1]^n$, а также определена функция, такая что
\begin{equation*}
    f = \begin{cases}
        1,& \forall i = \overline{1,\ldots, n}\,\, x_i\in \mathbb{Q}\\
        0,&\text{иначе}
    \end{cases}
\end{equation*}

\proof $\forall \T$ можно выбрать $\xi_i\in \mathbb{Q}$, тогда для такой пары $(\T, \overline{\xi})$:

\begin{equation*}
    \sigma(f, \T, \overline{\xi}) = \sum_{i=1}^k1\cdot|I_i| = |I| = 1
\end{equation*}

В то же время, $\forall \T$ можно выбрать $\xi_i\notin \mathbb{Q}$, тогда для такой пары $(\T, \hat{\xi})$:
\begin{equation*}
    \sigma(f, \T, \hat{\xi}) = \sum_{i=1}^k0\cdot|I_i| = 0 \Longrightarrow f\notin\mathcal{R}(I)
\end{equation*}

\subsection{Вычисление многомерного интеграла}

Вычислите интеграл
$$\iint\limits_{\substack{0\leq x\leq 1\\ 0\leq y\leq 1}}xy\d{x}\d{y}$$
рассматривая его как представление интегральной суммы при сеточном разбиении квадрата $$I = [0, 1]\times[0, 1]$$ на ячейки — квадраты со сторонами, длины которых равны $\frac{1}{n}$, выбирая в качестве точек $\xi_i$ нижние правые вершины ячеек

\begin{minipage}{0.5\textwidth}
Имеется функция $f = xy,\ |I| =\displaystyle\frac{1}{n^2}$
\begin{equation*}
    \begin{aligned}
        \sigma(f, \T, \xi) &= \sum_{i=1}^n \sum_{j=0}^{n-1}\frac{i}{n}\cdot\frac{j}{n}\cdot\frac{1}{n^2}
        &= \frac{1}{n^4}\sum_{i=1}^n\sum_{j=0}^{n-1} i\cdot j
        &= \frac{1}{n^4}\sum_{i=1}^n i \sum_{j=0}^{n-1} j
        &= \frac{n (n-1) }{n^4} \sum_{i=1}^ni
        &= \frac{n^2 (n+1) (n-1)}{4n^4}
        % \underset{n\to\infty}{\longrightarrow}\frac{1}{4}
    \end{aligned}
\end{equation*}
Заметим, что $\lim\limits_{n\rightarrow\infty}\displaystyle \frac{n^2 (n+1) (n-1)}{4n^4}=\frac{1}{4}$
\end{minipage}

\begin{center}
    % \documentclass[a4paper,10pt]{article}
% \usepackage[english,russian]{babel}
% \usepackage{tikz}

% \begin{document}
 \begin{tikzpicture}[scale=2]
\draw[step=0.25cm, gray, very thin] (0,0) grid (2,2);

\draw[thick] (0,0) rectangle (2,2);

\node at (-0.1, -0.1) {$0$};
\node at (1, -0.3) {$n$ штук};
\node at (2.1, -0.1) {$1$};
\node at (-0.1, 2) {$1$};

\foreach \i in {0.25, 0.5, ..., 2} {
    \foreach \j in {0, 0.25, ..., 1.75} {
        \fill (\i,\j) circle (1pt);
    }
}

\node[align=center] at (0.9,-0.6) {Рисунок того как мы выбираем в примере точки на разбиение};

\end{tikzpicture}
% \end{document}

\end{center}
