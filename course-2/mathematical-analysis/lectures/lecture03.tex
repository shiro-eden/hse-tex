\section{Лекция 3}
\definition Пусть имеется $M\subset\mathbb{R}^n$. Точку $x_0\in M$ будем называть \textit{внутренней} точкой $M$, если $$\exists\ve>0:B_{\ve}(x_0)\subset M$$

\definition Точку $x_0\in M$ будем называть \textit{внешней} точкой $M$, если $$\exists\ve>0:B_{\ve}(x_0)\subset (\mathbb{R}^n\setminus M)$$

\ex $M=[0;1)$. тогда
\begin{equation*}
    \begin{cases}
        x=0.5&\text{ — внутренняя}\\
        x=0&\text{ — не внутренняя}\\
        x=2&\text{ — внешняя}
    \end{cases}
\end{equation*}

\definition Точку $x_0\in\mathbb{R}^n$ будем называть \textit{граничной} точкой $M$, если $$\forall \ve>0:\ \left(B_{\ve}(x_0)\cap M\right)\ne\varnothing\wedge B_{\ve}(x_0)\cap(\mathbb{R}^n\setminus M)\ne\varnothing$$

\mark $\partial M$ — множетсво всех граничных точек $M$

\begin{center}
    \begin{tikzpicture}[
    set/.style={dashed, thick},
    arrow/.style={-{Stealth[scale=1.2]}, thick}
]

\begin{scope}[scale=1.8]


\draw[set, red!] (-1.7,0) to[out=130,in=70] (-2.9,0.3)
           to[out=-100,in=60] (-2.8,-1.)
           to[out=-120,in=-100] (-1.8,-1.2)
           to[out=60,in=-50] cycle;
\fill [pattern = {Lines[angle = -45, line width = 1pt, distance = 10pt]},
    pattern color = blue]
        (-1.7,0) to[out=130,in=70] (-2.9,0.3)
           to[out=-100,in=60] (-2.8,-1.)
           to[out=-120,in=-100] (-1.8,-1.2)
           to[out=60,in=-50] cycle;


\draw[thin, red!] (0,0) to[out=30,in=150] (0.5,0.3)
           to[out=-30,in=60] (1.,-1.)
           to[out=-120,in=-30] (0.3,-1.2)
           to[out=120,in=-120] (-0.3,-0.2)
           to[out=60,in=180] cycle;
\fill [pattern = {Lines[angle = -45, line width = 1pt, distance = 10pt]},
    pattern color = blue]
        (0,0) to[out=30,in=150] (0.5,0.3)
           to[out=-30,in=60] (1.,-1.)
           to[out=-120,in=-30] (0.3,-1.2)
           to[out=120,in=-120] (-0.3,-0.2)
           to[out=60,in=180] cycle;


\fill[green!] (-1, 0.4) circle(1.25pt);
\fill[green!] (-0.4, -0.7) circle(1.25pt);
\fill[green!] (-1.2, -0.5) circle(1.25pt);
\fill[green!] (-0.9, -1.2) circle(1.25pt);
\fill[green!] (-0.7, -0.1) circle(1.25pt);
\fill[green!] (-3.2, -0.8) circle(1.25pt);
\fill[green!] (-3.4, 0.1) circle(1.25pt);
\fill[green!] (1.5, -0.3) circle(1.25pt);

\end{scope}


\node[align=center] at (-1.5,-4) {Синяя область - пример множества внутренних точек \\ Зеленые - пример внешних (изображена часть точек) \\ Красные границы -  пример множества граничных точек (пример $\partial M$)};


\end{tikzpicture}

\end{center}


\ex $M=[0;1)\Longrightarrow x=0;1$ — граничные

\definition Точку $x_0\in M$ будем называть \textit{изолированной} точкой $M$, если $$\exists \ve>0:\ \stackrel{\circ}{B_{\ve}}(x_0)\cap M=\varnothing$$

\ex $M=[0;1]\cup \{3\}\Longrightarrow x=3$ — изолированная

\definition Точку $x_0\in\mathbb{R}^n$ будем называть \textit{предельной} точкой $M$, если $$\forall \ve>0:\ \stackrel{\circ}{B_{\ve}}(x_0)\cap M\ne\varnothing$$

\comment Из определения следует, что изолированные точки не являются предельными

\definition Точку $x_0\in\mathbb{R}^n$ будем называть \textit{точкой прикосновения} $M$, если $$\forall \ve>0:\ B_{\ve}(x_0)\cap M\ne\varnothing$$

\comment Точки прикосновения = изолированные точки $\oplus$ предельные точки

\begin{center}
    
\begin{tikzpicture}[
    set/.style={dashed, thick},
    arrow/.style={-{Stealth[scale=1.2]}, thick}
]
\begin{scope}[scale=0.8]

    % --- ЗЕЛЕНАЯ фигура (слева) ---
    \fill [pattern = {Lines[angle = -45, line width = 1pt, distance = 10pt]},
    pattern color = green]
        (-6,0) to[out=30,in=150] (-3,1)
               to[out=-30,in=90] (-2,-1)
               to[out=-90,in=0] (-4,-2)
               to[out=180,in=-120] cycle;

    % Контур
    \draw[set, green!]
        (-6,0) to[out=30,in=150] (-3,1)
               to[out=-30,in=90] (-2,-1)
               to[out=-90,in=0] (-4,-2)
               to[out=180,in=-120] cycle;

    % Красные точки
    \fill[red!] (-7, 0) circle (2pt);
    \fill[red!] (-6, -2) circle (2pt);
    \fill[red!] (-1.5, 0) circle (2pt);

    % Подпись под зеленой
    \node[align=center] at (-4,-3.5) {Пример точек \\ Красные - изолированные. Зелёные - предельные};


    % --- СИНЯЯ фигура (справа) ---
    \fill [pattern = {Lines[angle = -45, line width = 1pt, distance = 10pt]},
    pattern color = blue]
        (3,0) to[out=30,in=150] (6,1)
              to[out=-30,in=90] (7,-1)
              to[out=-90,in=0] (5,-2)
              to[out=180,in=-120] cycle;

    % Контур
    \draw[set, blue!]
        (3,0) to[out=30,in=150] (6,1)
              to[out=-30,in=90] (7,-1)
              to[out=-90,in=0] (5,-2)
              to[out=180,in=-120] cycle;

    \fill[blue!] (2, 0) circle (2pt);
    \fill[blue!] (3, -2) circle (2pt);
    \fill[blue!] (7.5, 0) circle (2pt);

    % Подпись под синей
    \node[align=center] at (5,-3.5) {Синие точки - точки прикосновения};
\end{scope}
\end{tikzpicture}

\end{center}


\definition Множество всех точек прикосновения $M$ называется \textit{замыканием} $M$ и обозначается как $\overline {M}$

\ex $M=(0;1)\cup(1;2]\Longrightarrow\overline{M}=[0;2]$

\ex $M=\{x\in[0;1]\colon x\in \mathbb{Q}\}\Longrightarrow\overline{M}=[0;1]$

\definition Множество $M\subset\mathbb{R}^n$ называется \textit{открытым}, если все его точки внутренние

\definition Множество $M\subset R^n$ называется замкнутым, если $\mathbb{R}^n\setminus M$ — открыто

\ex $\begin{cases}
    (0;1)&\text{ — открыто в $\mathbb{R}$}\\
    [0;1]&\text{ — замкнуто, т.к. $(-\infty;0)\cup(1;+\infty)$ открыто в $\mathbb{R}$}\\
    [0;1)&\text{ — ни открыто, ни замкнуто в $\mathbb{R}$}
\end{cases}$

\definition Множество $K\subset\R^n$ называется \textit{компактом}, если из $\forall$ его покрытия открытыми множествами можно выделить конечное подпокрытие

\comment Если хотя бы для какого-то покрытия это не выполняется, то $K$ — не компакт

\ex Пусть $M=(0,1)$ покроем $\left\{A_n=\left(0;1-\frac{1}{n}\right)\right\}_{n=1}^\infty$

При $n\rightarrow\infty$ $M\subset \displaystyle\bigcup_{n=1}^\infty A_n$, но $\forall$ фиксированного $N$: $M\not\subset\displaystyle\bigcup_{n=1}^{\infty}\Longrightarrow$ не компакт

\definition Множество $M\subset \mathbb{R}^n$ — называется \textit{ограниченным}, если $$\exists x_0\in\mathbb{R}^n\text{ и }\exists r>0\text{, такой что }M\subset B_{r}(x_0)$$

\subsection{Критерий замкнутости}
% Наличие социофобии

\theorem $M$ — замкнуто $\Longleftrightarrow$ $M$ содержит \textbf{все} cвои предельные точки

\proof Докажем необходимость и достаточность
\begin{enumerate}
    \item \textit{(Необходимость)} Докажем $\Longrightarrow$ от противного
    \begin{itemize}
        \item Пусть $x_0$ — предельная для $M$ и $x_0\notin M$. Тогда, $\forall\ve>0\ \stackrel{\circ}{B_{\ve}}(x_0)\cap M\ne\varnothing\text{ и }x_0\in\mathbb{R}^n$
        \item По условию $M$ — замкнуто, то есть $\mathbb{R}^n\setminus M$ — открыто $\Longrightarrow$ все его точки внутренние и $\exists r>0$:
        $$B_{r}(x_0)\subset\mathbb{R}^n\setminus M\Longrightarrow\stackrel{\circ}{B_r(x_0)}\subset\mathbb{R}^n\setminus M\text{ и }\stackrel{\circ}{B_r}(x_0)\cap M=\varnothing$$

        Пришли к противоречию $\Longrightarrow$ $M$ содержит все свои предельные точки\qed
    \end{itemize}
    \item \textit{(Достаточность)} Докажем $\Longleftarrow$

    Пусть $y_0$ — не является предельной для $M$, то есть $y_0\in\mathbb{R}^n\setminus M\Longrightarrow\exists r>0$:
    \begin{equation*}
        \begin{cases}
            \stackrel{\circ}{B_{r}}(y_0)\cap M=\varnothing\\
            y_0\in\mathbb{R}^n\setminus M
        \end{cases}\Longrightarrow B_r(y_0)\subset \mathbb{R}^n\setminus M
    \end{equation*}
    $\Longrightarrow\mathbb{R}^n\setminus M$ — открытое и состоит из всех точек, не являющихся предельными $\Longrightarrow$ $M$ — замкнуто по определению\qed
\end{enumerate}

\newpage
