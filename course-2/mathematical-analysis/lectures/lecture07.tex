% \documentclass[a4paper,10pt]{article}
% \usepackage{../header}
%
% \usepackage[utf8]{inputenc}
%
% \begin{document}

\section{Лекция 7}

\subsection{Интеграл Дарбу как предел сумм Дарбу}
\theorem Пусть $I\subset \R^n$ — замкнутый брус, а $f: I \mapsto \R$ — ограничена. Тогда:
\begin{equation*}
    \oi = \lim_{\Delta_{\T}\to0}\os(f, \T) \qquad \text{и} \qquad \ui = \lim_{\Delta_{\T} \to 0} \us(f, \T)
\end{equation*}

\proof Докажем, что $\ui = \displaystyle\lim_{\Delta_{\T} \to 0} \us(f, \T) \quad (= \displaystyle\sup_{\T} \us (f, \T))$
\begin{enumerate}
    \item $f$-ограничена на $I \implies \exists C > 0: \forall x \in I\quad |f(x)| \leqslant C$
    \item т.к. по определению $\underline{I} = \displaystyle\sup_{\T}\us(f, \T)$, то $\forall \ve > 0 \,\, \exists\T_1 = \{I_i^1\}_{i=1}^{m_1}:\ \ui-\ve < \us(f, \T_1) \leqslant \ui < \ui + \ve$
    \item Пусть $G = \displaystyle\bigcup_{i=1}^{m_1}\partial I_i^1$ - объединение границ брусов $I^1_i \in \T_1$ (без повторов). Тогда $G$ множество меры нуль по Лебегу (т.к. границы --- мн-ва меры нуль по Лебегу)
    \item Пусть $\T_2$ - произвольное разбиение $I: \,\, \T_2 = \{I_i^2\}_{i=1}^{m_2}$ \\
    Рассмотрим два множества брусов:\\
    \begin{equation*}
    \begin{aligned}
        A = \{I_i^2 \in \T_2: I_i^2 \cap G \ne \varnothing\} \qquad \text{и} \qquad B = \T_2\backslash A \implies\\
        \forall \ve > 0 \,\, \exists \delta(\ve) > 0 : \forall \T_2: \Delta_{\T_2} < \delta \text{ верно, что } \sum_{I_i^2 \in A} |I_i^2| < \epsilon
    \end{aligned}
    \end{equation*}
    т.к. наши брусочки $I^2_i$ по построению лежат в $G$, а по 3 пункту оно множество меры нуль.

    \begin{center}
        


\begin{tikzpicture}[scale=1.9, every node/.style={inner sep=0,outer sep=0},
    set/.style={dashed, thick},
    arrow/.style={-{Stealth[scale=1.2]}, thick}]
    \begin{scope}
        \draw[thick] (0, 0) rectangle (1, 1);
        \draw[thick][blue] (0.5, 0) -- (0.5, 1);
        \draw[thick][blue] (0, 0.5) -- (1, 0.5);
        \node at (0.53, -0.3) {разбиение $T_1$};
    \end{scope}




    \begin{scope}[xshift=2cm]
        \draw[thick] (0, 0) rectangle (1, 1);
        \draw[thick] (0.5, 0) -- (0.5, 1);
        \draw[thick] (0, 0.5) -- (1, 0.5);

        \fill[pattern = {Lines[angle = -45, line width = 1pt, distance = 5pt]},
    pattern color = blue,
    even odd rule]
            (0, 0) rectangle (0.5, 0.5)
            (0.05, 0.05) rectangle (0.45, 0.45)

            (0.5, 0) rectangle (1, 0.5)
            (0.55, 0.05) rectangle (0.95, 0.45)

            (0.5, 0.5) rectangle (1, 1)
            (0.55, 0.55) rectangle (0.95, 0.95)

            (0, 0.5) rectangle (0.5, 1)
            (0.05, 0.55) rectangle (0.45, 0.95);
        \draw[set][blue]
            (0.05, 0.05) rectangle (0.45, 0.45)
            (0.55, 0.55) rectangle (0.95, 0.95)
            (0.05, 0.55) rectangle (0.45, 0.95)
            (0.55, 0.05) rectangle (0.95, 0.45);

        \node at (0.53, -0.3) {Граница $G$ бруса $T_1$};
    \end{scope}



    \begin{scope}[xshift=4cm]
        \draw[thick] (0, 0) rectangle (1, 1);
        \draw[thick][red] (0.1, 0) -- (0.1, 1);
        \draw[thick][red] (0.25, 0) -- (0.25, 1);
        \draw[thick][red] (0.4, 0) -- (0.4, 1);
        \draw[thick][red] (0.6, 0) -- (0.6, 1);
        \draw[thick][red] (0.9, 0) -- (0.9, 1);
        \draw[thick][red] (0, 0.4) -- (1, 0.4);
        \draw[thick][red] (0, 0.6) -- (1, 0.6);
        \draw[thick][red] (0, 0.1) -- (1, 0.1);
        \draw[thick][red] (0, 0.9) -- (1, 0.9);
        \node at (0.53, -0.3) {Какое-то разбиение $T_2$};
    \end{scope}

    \begin{scope}[xshift=0.2cm, yshift=-2cm]
        \draw[thick] (0, 0) rectangle (1, 1);
        \draw[thick] (0.5, 0) -- (0.5, 1);
        \draw[thick] (0, 0.5) -- (1, 0.5);

        \fill[pattern = {Lines[angle = -45, line width = 1pt, distance = 5pt]},
    pattern color = blue,
    opacity=0.3,
    even odd rule]
            (0, 0) rectangle (0.5, 0.5)
            (0.05, 0.05) rectangle (0.45, 0.45)

            (0.5, 0) rectangle (1, 0.5)
            (0.55, 0.05) rectangle (0.95, 0.45)

            (0.5, 0.5) rectangle (1, 1)
            (0.55, 0.55) rectangle (0.95, 0.95)

            (0, 0.5) rectangle (0.5, 1)
            (0.05, 0.55) rectangle (0.45, 0.95);
        \draw[set][blue][opacity=0.3]
            (0.05, 0.05) rectangle (0.45, 0.45)
            (0.55, 0.55) rectangle (0.95, 0.95)
            (0.05, 0.55) rectangle (0.45, 0.95)
            (0.55, 0.05) rectangle (0.95, 0.45);



        \draw[thick] (0, 0) rectangle (1, 1);
        \draw[thick][red][opacity=0.5]
            (0.1, 0) -- (0.1, 1)
            (0.25, 0) -- (0.25, 1)
            (0.4, 0) -- (0.4, 1)
            (0.6, 0) -- (0.6, 1)
            (0.9, 0) -- (0.9, 1)
            (0, 0.4) -- (1, 0.4)
            (0, 0.6) -- (1, 0.6)
            (0, 0.1) -- (1, 0.1)
            (0, 0.9) -- (1, 0.9);
        \node[align=center] at (0.53, -0.3) {Как прошлые разбиения и граница $G$ \\ выглядят на одном рисунке};
    \end{scope}



    \begin{scope}[xshift=3.8cm, yshift=-2cm]
        \draw[thick] (0, 0) rectangle (1, 1);

        \fill[pattern = {Lines[angle = -45, line width = 1pt, distance = 5pt]},
    pattern color = red,
    opacity=0.3,
    even odd rule]
            (0, 0) rectangle (0.5, 0.5)
            (0.05, 0.05) rectangle (0.45, 0.45)

            (0.5, 0) rectangle (1, 0.5)
            (0.55, 0.05) rectangle (0.95, 0.45)

            (0.5, 0.5) rectangle (1, 1)
            (0.55, 0.55) rectangle (0.95, 0.95)

            (0, 0.5) rectangle (0.5, 1)
            (0.05, 0.55) rectangle (0.45, 0.95);
        \draw[set][red][opacity=0.7]
            (0.05, 0.05) rectangle (0.45, 0.45)
            (0.55, 0.55) rectangle (0.95, 0.95)
            (0.05, 0.55) rectangle (0.45, 0.95)
            (0.55, 0.05) rectangle (0.95, 0.45);

        \draw[arrow][red] (-0.3, 0.5) -- (0.05, 0.51);
        \node at (-0.4, 0.5) {$A$};



        \fill[pattern = {Lines[angle = 55, line width = 1pt, distance = 5pt]},
    opacity=0.5]
            (0.05, 0.05) rectangle (0.45, 0.45)

%             (0.5, 0) rectangle (1, 0.5)
            (0.55, 0.05) rectangle (0.95, 0.45)

%             (0.5, 0.5) rectangle (1, 1)
            (0.55, 0.55) rectangle (0.95, 0.95)

%             (0, 0.5) rectangle (0.5, 1)
            (0.05, 0.55) rectangle (0.45, 0.95);


        \draw[arrow] (1.3, 0.5) -- (0.8, 0.7);
        \node at (1.4, 0.5) {$B$};

        \node[align=center] at (0.53, -0.3) {Как выглядят множества $A$ и $B$};
    \end{scope}

\end{tikzpicture}

    \end{center}


    \item С другой стороны $\forall I_i^2 \in B$ верно, что $I_i^2 \in \T_1 \cap \T_2$
\end{enumerate}

Хотим рассмотреть
\begin{equation*}
\begin{aligned}
    |\ui - \us(f, \T_2)| = |I-\us(f, \T_1\cap \T_2) + \us(f, \T_1\cap \T_2) -\us(f, \T_2)| &\leqslant \underbrace{|I-\us(f, \T_1\cap \T_2)|}_* + \underbrace{|\us(f, \T_1\cap \T_2) -\us(f, \T_2)|}_{**} \\
    &< \ve + 2 C \ve = \ve(1 + 2C)
\end{aligned}
\end{equation*}\qed

\begin{itemize}
    \item[*] из пункта 2: $\ui - \ve < \us(f, \T_1) \leqslant \us(f, \T_1\cap \T_2) \leqslant \ui < \ui + \ve \implies |\ui - \us(f, \T_1\cap\T_2)| < \ve$\\
    \item[**] Пояснение ниже \begin{equation*}
        \begin{aligned}
            \left|\us(f, \T_1\cap\T_2) - \us(f, \T_2)\right| &= \left|\sum_{I_i^2\in B}m_i|I_i^2| + \sum_{I_i\in \T_1\cap A}m_i|I_i^2| - \sum_{I_i^2\in B}m_i|I_i^2| - \sum_{I_i^2\in A}m_i|I_i^2|\right| ~~~ \textit{\text{Переход с равном по пункту 5}} \\
            &\leqslant \left|\sum_{I_i\in \T_1\cap A}m_i|I_i^2|\right| + \left|\sum_{I_i^2\in A}m_i|I_i^2|\right|\\
            &\leqslant 2\left|\sum_{I_i^2\in A}m_i|I_i^2|\right| ~~~ \textit{\text{Следующий переход по пункту 1}} \\
            &\leqslant 2C\left|\sum_{I_i^2\in A}|I_i^2|\right| ~~~ \textit{\text{Следующий переход по пункту 4}} \\
            & < 2C\ve
        \end{aligned}
    \end{equation*}
\end{itemize}


\subsection{Критерий Дарбу интегрируемости функции по Риману}
$I\in\mathbb{R}^n\text{ — замкнутый брус, } f:I\mapsto \mathbb{R}, f\in \mathcal{R}(I)\Longleftrightarrow f$ — ограничена на $I$ и $\ui=\oi$

\proof Необходимость
\begin{itemize}
    \item $f\in\riman{I}\Longrightarrow$ по необходимому условию интегрируемости функции по Риману на замкнутом брусе, $f$ — ограничена на $I$
    \item Покажем, что $\underline{\mathcal{I}}=\mathcal{I},\overline{\mathcal{I}}=\mathcal{I}\Longrightarrow\underline{\mathcal{I}}=\overline{\mathcal{I}}$

    \begin{enumerate}
        \item $f\in\riman{I}\Longrightarrow\forall \ve >0\ \exists \delta >0\ \forall(\mathbb{T},\xi):\Delta_{\mathbb{T}}<\delta \ |\sigma(f,\mathbb{T},\xi)-\mathcal{I}|<\ve$
        \item $\ui=\sup\limits_{\mathbb{T}}\us(f,\T)=\lim\limits_{\Delta\rightarrow 0}\us(f,\T) \Longrightarrow |\ui-\us|<\ve$

        $\forall \ve>0\ \exists\delta\ \exists\mathbb{T}:\Delta_{\mathbb{T}}<\delta: |\ui-\us|<\ve$
        \item $\us(\mathbb{T},\xi)=\inf\limits_{\xi}\sigma(f,\mathbb{T},\xi)$

        $\forall\mathbb{T},\ \forall \ve>0\ \exists \xi:|\us-\sigma|<\ve$
    \end{enumerate}
\end{itemize}

$|\mathcal{I}-\ui|\leqslant|\mathcal{I}-{\ui}-\sigma+\sigma+{\us}-\us|\leqslant|\mathcal{I}-\sigma|+|\ui-\us|+|\sigma-\us|<3\ve$\qed

\proof Достаточность

$f$ — ограничена и $\ui=\oi$. Имеем
\begin{equation*}
    \us(f,\mathcal{T})=\inf\limits_{\xi}\leqslant\sigma(f,\mathbb{T},\xi)\leqslant \sup\limits_{\xi}(f,\mathbb{T},\xi)=\os(f,\mathbb{T})
\end{equation*}

Тогда, при $\lim\limits_{\Delta_{\T}\rightarrow 0} \us=\ui,\ \lim\limits_{\Delta_{\T}\rightarrow0}\os=\oi$ получаем $\ui=\oi$(Условие ограниченнсоти $f$ даёт нам возможность применять неравенство выше)\qed


\subsection{Интегрирование по допустимым множествам}

\definition Множество $D\subset\mathbb{R}^n$ называется \textit{допустимым}, если

\begin{itemize}
    \item $D$ — ограниченно
    \item $\partial D$ — множество меры нуль по Лебегу
\end{itemize}

\ex Допустимого и не допустимого множества\begin{enumerate}
        \item $D_1 = (0, 1)$
        \begin{itemize}
            \item ограничено - да
            \item $\partial D_1 = \{0\} \cup \{1\}$ - мн-во меры нуль - да
        \end{itemize}
        $D_1$ - допустимое множество

        \item $D_2 = [0, 1] \cap \QQ$
        \begin{itemize}
            \item ограничено - да
            \item $\partial D_2 = [0, 1]$ - мн-во меры нуль - нет
        \end{itemize}
        $D_2$ - не допустимое множество
    \end{enumerate}


\definition Пусть $D\subset\mathbb{R}^n, f:D\rightarrow\mathbb{R}$. Тогда, интегралом Римана $f$ по $D$ называется число $\mathcal{I}$:
\begin{equation*}
    \mathcal{I}=\int\limits_{D}f(\overline{x})\d{\overline{x}}=\int\limits_{I\supset D}f\cdot \chi_{D}(\overline{x})\d{\overline{x}}\text{, где }\chi_{D}=\begin{cases}
        1,\overline{x}\in D\\
        0,\overline{x}\in D
    \end{cases}
\end{equation*}

Если $\mathcal{I} < \infty$, то $f \in\riman{D}$

\begin{center}
    
\begin{tikzpicture}
    % Рисуем полосатый прямоугольник
    \fill[pattern=north east lines, opacity=0.4] (1,0) rectangle (5,3);

    % Определяем гладкую кляксу (неправильной формы)
    \begin{scope}
        \clip (2,1.5) .. controls (2.5,2.5) and (3.5,2.8) .. (4,2)
              .. controls (4.3,1.5) and (4,0.8) .. (3.5,0.5)
              .. controls (2.8,0.2) and (1.8,0.7) .. (2,1.5) -- cycle;
        \fill[white] (0,0) rectangle (6,4); % Закрашиваем кляксу белым
    \end{scope}
    \node at (1.7, 2.5) {$I$};

    % Контуры для ясности
    \draw (1,0) rectangle (5,3);
    \draw (2,1.5) .. controls (2.5,2.5) and (3.5,2.8) .. (4,2)
          .. controls (4.3,1.5) and (4,0.8) .. (3.5,0.5)
          .. controls (2.8,0.2) and (1.8,0.7) .. (2,1.5) -- cycle;
    \node at (3, 1.5) {$D$};

    \node[align=center] at (2.9, -1) {Закрашенная область не вносит вклад в объем \\ так как $f(x)\cdot\chi_D=0$};
\end{tikzpicture}

\end{center}

