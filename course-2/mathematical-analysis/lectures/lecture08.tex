% \documentclass[a4paper,10pt]{article}
% \usepackage{../header}
% \begin{document}


\section{Лекция 8}
\subsection{Интегрирование по допустимым множествам(Продолжение)}
\textbf{Корректность определения допустимых множеств.} Пусть $D \subset I_1 \subset \R^n, D \subset I_2 \subset \R^n$ - замкнутые брусы, тогда
\begin{equation*}
    \int\limits_{I_1} f\cdot\chi_D\d{x}\text{ и }\int\limits_{I_2}f\cdot\chi_{D}\d{x}
\end{equation*}

либо существуют и равны, либо оба не существуют вообще


\begin{center}
    
\begin{tikzpicture}[scale=1.75]

    \draw (0, 0) rectangle (3, 2);
    \node at (0.4, 0.4) { $I_1$ };

    \draw (1.5, 1) rectangle (5, 2.5);
    \node at (4.5, 2.1) { $I_2$ };


    \draw[thin] (2, 1.8) to[out=0,in=110] (2.8,1.5)
           to[out=-50, in=-30] (2.4,1.2)
           to[out=110, in=-70] (1.8,1.1)
           to[out=110, in=-180] cycle;
    \node at (1.67, 1.8) { $I$ };
    \node at (2.3, 1.5) { $D$ };

    \node[align=center] at (2.3, -0.3) {Как выглядят наши множества $I_1, I_2, I, D$};

\end{tikzpicture}

\end{center}



\proof Введем $I = I_1 \cap I_2 \supset D$, $I$ не пустое по построению. Покажем существование
\begin{itemize}
    \item $f\cdot\chi_D\in\riman{I_1}\Longrightarrow$ по критерию Лебега $f\cdot\chi_D$ ограничена на $I_1\Longrightarrow$ $f\cdot\chi_D$ ограничена на $D\Longrightarrow f$ ограничена на $D\Longrightarrow f\cdot\chi_D$ ограничена на $I_2$
    \item $f\cdot\chi_D\in\riman{I_1}\Longrightarrow$ по критерию Лебега $f\cdot\chi_D$ непрерывна почти всюду на $I_1\Longrightarrow f\cdot\chi_D$ непрерынва почти всюду на $D\Longrightarrow $ в худшем случае для $f\cdot\chi_D$ на $I_2$ добавятся разрывы на $\partial D\Longrightarrow f\cdot\chi_D$ непрерынва почти всюду на $I_2$
    \item Тогда, $f\cdot\chi_D\in\riman{I_1}\Longleftrightarrow f\cdot\chi_D\in\riman{I_2}$
\end{itemize}

Покажем равенство
\begin{itemize}
    \item Пусть $\mathbb{T}_i$ — разбиение на $I_i:\mathbb{T}_1$ и $\mathbb{T}_2$ совпадают на $I$
    \item Пусть $\xi^i$ — отмеченные точки для $\T_i$
    \item $\sigma(f\chi_D,\mathbb{T}_1,\xi^1)=\sum_{j}f\chi_D(\xi^1_j)|I_j^1|=\sum_j f(\xi^1_j)|I^1_j|=\sum_j f(\xi^2_j)|I^2_j|=\sum_j f\chi_D(\xi_j^2)|I_j^2|=\sigma(f\chi_D, \mathbb{T}_2, \xi^2)$
\end{itemize}\qed

\comment Все свойства интеграла Римана и критерия Лебега для бруса справедливы и для других допустимых множеств

\subsection{Теорема Фубини}
Пусть имеются $I_x\subset\mathbb{R}^n, I_y\subset\mathbb{R}^m, I_x\times I_y\subset \mathbb{R}^{m+n}$ — замкнутые брусы, $f:I_x\times I_y\rightarrow \mathbb{R}$, $f\in\riman{I_x\times I_y}$ и $\forall$ фиксированного $x\in I_x \implies f(x,y)\in\riman{I_y}\Longrightarrow$
\begin{equation*}
    \int\limits_{I_x\times I_y} f(\overline{x}, \overline{y})\d{\overline{x}}\d{\overline{y}}=\int\limits_{I_x}\left(\int\limits_{I_y}f(\overline{x},\overline{y})\d{\overline{y}}\right)\d{\overline{x}}=\int\limits_{I_x}\d{\overline{x}}\int\limits_{I_y}f(\overline{x}, \overline{y})\d{\overline{y}}
\end{equation*}

\comment аналагочино, если взять для $\forall$ фиксированного $y\in I_y$

\proof Воспользуемся тем, что $f\in\riman{I_x\times I_y}, \ f\in\riman{I_y}$, а также Критерием Дарбу
\begin{itemize}
    \item $\mathbb{T}_x=\{I_i^x\}$ — разбиение на $I_x$, $\mathbb{T}_y=\{I_j^y\}$ — разбиение на $I_y$, $\mathbb{T}_{x,y}=\{I_i^x\times I^y_j\}=\{I_{ij}\}$ — разбиение на $I_x\times I_y$, и при этом верно $|I_i^x| \cdot |I_j^y| = |I_{ij}|$
    \item \begin{equation*}
        \begin{aligned}
            \us(f,\mathbb{T}_{x,y})=\sum_{i,j} \inf\limits_{(x,y)\in I_{ij}} f(x,y)|I_{ij}|& \underset{\text{рис. ниже}}\leqslant \sum_{i,j} \inf\limits_{x\in I_i^x} \left(\inf\limits_{y\in I_j^y} f(x,y) \cdot |I_j^y|\right)|I_i^x|=\sum_i \inf\limits_{I^x_i} \underbrace{\left(\sum_j \inf\limits_{I^y_j} f(x,y)|I_j^y|\right)}_{\us(f(y), \mathbb{T}_y)}|I_i^x|\\
            &\leqslant \sum_i \inf\limits_{I^x_i}\underbrace{\left(\int\limits_{I_y} f(x,y)\d{y}\right)}_{g(x)}|I_i^x| \leqslant \us(g(x),\mathbb{T}_x)\\
            % &\leqslant \os(g(x),\mathbb{T}_x)\leqslant\ldots\leqslant \os(f,\mathbb{T}_{x,y})
            &\leqslant \os(g(x),\mathbb{T}_x)
        \end{aligned}
    \end{equation*}

    $\us(f,\mathbb{T}_{x,y})\leqslant \us(g(x),\mathbb{T}_x)\leqslant\os(g(x), \mathbb{T}_x)\leqslant\os(f,\mathbb{T}_{x,y})\Longrightarrow\exists \oi=\lim\limits_{\delta\rightarrow0} \us(g(x),\mathbb{T}_x) = \int\limits_{I_x\times I_y} f(\overline{x}, \overline{y})\d{\overline{x}}\d{\overline{y}}$

    \comment Последний знак неравентсва, получен аналогичными действиями для длинного неравенства выше, просто развернув в обратную сторону знаки неравенства для $\sup$
\end{itemize}\qed

\begin{center}
    % \documentclass[tikz,border=3.14mm]{standalone}
% \usepackage{pgfplots}
% \usepackage[utf8]{inputenc}         % кодировка исходного текста
% \usepackage[english,russian]{babel}
\pgfplotsset{compat=1.18}

% \begin{document}
\begin{tikzpicture}[scale=0.5]
    \begin{axis}[
        view={30}{30},
        xlabel={$x$},
        ylabel={$y$},
        zlabel={$z$},
        grid=major,
        colormap/viridis,
        width=10cm,
        height=10cm,
        zmin=0, zmax=4,
        xmin=-2, xmax=2,
        ymin=-2, ymax=2,
        ]

        % Воронка, направленная вниз
        \addplot3[
        surf,
        domain=-2:2,
        domain y=-2:2,
        samples=40,
        opacity=0.7,
        ] {4 - 1/(0.3 + x^2 + y^2)};

        % Плоскость сечения (x = 0.5)
        \addplot3[
        surf,
        domain=-2:2,
        domain y=0:4,
        samples=2,
        opacity=0.3,
        color=gray,
        ] (0.5, x, y);

        % Линия пересечения
        \addplot3[
        thick,
        red,
        domain=-1.65:1.85,
        samples=30,
        smooth,
        ] (0.5, x, {4 - 1/(0.3 + 0.25 + x^2)});

        \addplot3[
        only marks,
        mark=*,
        mark size=2,
        color=blue
        ] coordinates {(0.5, 0, {4 - 1/(0.3 + 0.25 + 0^2)})};

        \addplot3[
        only marks,
        mark=*,
        mark size=2,
        color=blue
        ] coordinates {(0, 0, {4 - 1 / 0.3})};


    \end{axis}

    \node[align=center] at (4, -1.5) {Если зафиксировать какой-то $x$ и искать $\inf$ по $y$ \\ то он всегда будет больше, чем $\inf$ на всей области};
\end{tikzpicture}
% \end{document}

\end{center}


\ex Случай, где нельзя интегрировать по т. Фубини

Возьмем следущую функцию
\begin{equation}
f(x, y) = \frac{x^2 - y^2}{(x^2 + y^2)^2} \ \ \text{на} \ \ [-1;1] \times [-1;1]
\end{equation}

Она не интегрируема по Риману на данной области области, т.к. функция неограничена

\begin{equation}
\lim_{y \to 0} f(0, y) = \lim_{y \to 0} \frac{-y^2}{y^4} = -\infty
\end{equation}

Что будет, если мы не поверили и решили применить т. Фубине? Вычислим интеграл $\int\limits_0^1 \left( \int\limits_0^1 f(x, y) dy \right) dx$. Заметим следущее внесение под дифференциал

\begin{equation}
\frac{d}{dy} \left( \frac{y}{x^2 + y^2} \right) = \frac{x^2 + y^2 - 2y^2}{(x^2 + y^2)^2} = \frac{x^2 - y^2}{x^2 + y^2}
\end{equation}

Получаем, тогда
\begin{equation}
  \int\limits_0^1 \left( \int\limits_0^1 \frac{x^2 - y^2}{x^2 + y^2} dy \right) dx =
  \int\limits_0^1 \left( \int\limits_0^1 d\left({\frac{y}{x^2 + y^2}}\right) \right) dx =
  \int\limits_0^1 \frac{1}{1+x^2} dx = \frac{\pi}{4}
\end{equation}

Теперь сошлемся на то что $f(x, y) = -f(y, x)$ и получим что $\int\limits_0^1 \left( \int\limits_0^1 f(x, y) dx \right) dy = -\frac{\pi}{4}$




\subsection{Теорема о замене переменных в кратном интеграле}
\theorem\textit{(Без доказательства)}
Пусть имеется $M_1,M_2\in\mathbb{R}^n$ — открытые множества. $\varphi:M_1\longrightarrow M_2$ — биективно, $\varphi,\varphi^{-1}$ — непрерывно дифференцируемые отображения

$D:\overline{D}\subset M_1$ — допустимое множество

$f:\varphi(D)\longrightarrow\mathbb{R}$

$f\in\riman{\varphi(D)}\Longleftrightarrow f(\varphi(t))\cdot|\det J_{\varphi}(t)|\in\riman{D}$ и
\begin{equation*}
    \int\limits_{\varphi(D)}f(x)\d{x}=\int\limits_{D}f(\varphi(t))\cdot|\det J_{\varphi}(t)|\d{t},\text{ где } J=\begin{pmatrix}
        \frac{\partial\varphi_1}{\partial t_1}&\ldots&\frac{\partial\varphi_1}{\partial t_n}\\
        \vdots&\ddots&\vdots\\
        \frac{\partial\varphi_n}{\partial t_1}&\ldots&\frac{\partial\varphi_n}{\partial t_n}
    \end{pmatrix}
\end{equation*}

\comment $(x_1,\ldots,x_n)\overset{\varphi}{\longrightarrow}(t_1,\ldots,t_n)$, где $x_i=\varphi_i(t_1,\ldots,t_n)$

\ex Ранее мы переходили к полярным координатам так: $(x,y)\rightarrow(r,\varphi)$, при этом $\begin{cases}
    x=r\cos{\varphi}\\
    y=r\sin{\varphi}
\end{cases}$

$J=\begin{pmatrix}
    \cos{\varphi} & -\sin{\varphi}\cdot r\\
    \sin{\varphi} & \cos{\varphi}\cdot r
\end{pmatrix}$

$|J_{\varphi^{-1}}|=|J_{\varphi}|^{-1}$

\begin{center}
     % локализация и переносы
% \documentclass[tikz,border=10pt]{standalone}
% \usepackage[utf8]{inputenc}         % кодировка исходного текста
% \usepackage[english,russian]{babel}
\usetikzlibrary{decorations.pathmorphing}
% \usetikzlibrary{arrows.meta, patterns, patterns.meta}

% \begin{document}
\begin{tikzpicture}[scale=2, every node/.style={inner sep=0,outer sep=0},
    set/.style={dashed, thick},
    arrow/.style={-{Stealth[scale=1.2]}, thick}]

     \begin{scope}[shift={(0.3, 0.3)}]

        \draw[blue, thick] (0,0) -- (1,0);
        \draw[blue, thick] (0,0) -- (0,1);
        \draw[blue, thick] (0,1) -- (1,1);
        \draw[blue, thick] (1,0) -- (1,1);
    \end{scope}

     \begin{scope}

        \draw[->] (0, 0) -- (1.5, 0) node[right] {$x$};
        \draw[->] (0, 0) -- (0, 1.5) node[above] {$y$};
    \end{scope}



    \draw[arrow] (1.8, 0.5) -- (2.6, 0.5);
    \node at (2.2, 0.35) {$\varphi$};

    \draw[arrow] (2.6, 1) -- (1.8, 1);
    \node at (2.3, 1.15) {$\varphi^{-1}$};




    \begin{scope}[shift={(3.2, 0.3)}, rotate=10]

        \draw[blue, thick] (0,0) .. controls (0.5, 0.1) .. (1,0);
        \draw[blue, thick] (0,0) .. controls (0.1, 0.5) .. (0,1);
        \draw[blue, thick] (0,1) .. controls (0.5, 1.1) .. (1,1);
        \draw[blue, thick] (1,0) .. controls (1.1, 0.5) .. (1,1);
    \end{scope}


     \begin{scope}[shift={(3, 0)}, rotate=10]

        \draw[->] (0, 0) -- (1.5, 0) node[right] {$u$};
        \draw[->] (0, 0) -- (0, 1.5) node[above] {$v$};
    \end{scope}

     \begin{scope}[shift={(3, 0)}, opacity=0.3]

        \draw[->] (0, 0) -- (1.5, 0) node[right] {$x$};
        \draw[->] (0, 0) -- (0, 1.5) node[above] {$y$};
    \end{scope}


    \node[align=center] at (2.3, -0.3) {Отображение допустимого множества в новые координаты};

\end{tikzpicture}
% \end{document}



\end{center}



% \end{document}
